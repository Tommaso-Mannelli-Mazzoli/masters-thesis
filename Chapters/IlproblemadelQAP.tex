% !TeX spellcheck = en_US
\chapter{The quadratic assignment problem}
\label{Chap:TheQAP}

\start{I}{n this} chapter we will describe formulations and variants of the Quadratic Assignment Problem. There are several equivalent formulations of the QAP, each one exploiting a different feature of the structure of the problem. Different formulations lead to different solution approaches. Other formulations and variants can be found in \cite[Ch. 7]{Burkard2012}.



\section{Description and formulations}
\label{sec:formulations}
A set of $n$ facilities has to be allocated to a set of $n$ locations. We are given three matrices:
\begin{itemize}
	\item $\bm F\in\R^{n\times n}$, where $\bm F = (f_{ij})$ and $f_{ij}$ is the \textit{flow} between facility $i$ and facility $j$. Therefore, $f_{ij}$ could be viewed as the amount of supplies transported between the two facilities. Hence, it is the cost per unit length.
	\item $\bm D\in\R^{n\times n}$, where $\bm D  = (d_{rs})$ and $d_{rs}$ is the \textit{distance} between location $r$ and location $s$.  Note that $d_{rs}$ is a length.
	\item $\bm C\in\R^{n\times n}$, where $\bm C= (c_{ir})$ and $c_{ir}$ is the \textit{cost} of placing facility $i$ at location $r$. Note that $c_{i\pi(i)}$ is a cost.
\end{itemize}

\subsection{Combinatorial formulation}
The feasible set is the set of permutation of $n$ elements $\Sn$. 

If $\pi \in \Sn$, the product $f_{ij}\, d_{\pi(i)\pi(j)}$ is the transportation cost associated to assigning facility $i$ to location $\pi(i)$ and facility $j$ to location $\pi(j)$. That is, the transportation cost is given by the product flow times distances. 

Each term $c_{i \pi(i)}+\sum_{j=1}^n f_{ij}d_{\pi(i)\pi(j)}$ represents the total cost, related to facility $i$ given by the cost for installing it at location $\pi(i)$ plus the transportation costs to all facilities $j$, if installed at locations $\pi(1),\pi(2),\dots,\pi(n)$. 

Hence, the QAP can be written as 

\begin{tcolorbox}[title=Combinatorial formulation]
\begin{equation}
\label{eq:defQAP}
\min_{\pi \in \Sn}{ \left[\sum_{i=1}^n\sum_{j=1}^n f_{ij}d_{\pi(i)\pi(j)}+\sum_{i=1}^n c_{i\pi(i)}\right]}
\end{equation}
\end{tcolorbox}
In future, $z(\pi)$ will denote the quadratic term of the objective function
\begin{equation}
\label{eq:DefFunObj}
z(\pi)= \sum_{i=1}^n\sum_{j=1}^n f_{ij}d_{\pi(i)\pi(j)}.
\end{equation}

\noindent An instance of the \QAP	\ with input matrices $\bm F$, $\bm D$, $\bm C$ is denoted by $\QAP(\bm F, \bm D, \bm C)$. If there is no linear term (hence, $\bm C$ is the null matrix), we just write $\QAP(\bm F, \bm D)$.

\subsection{Lawler's general formulation}
A more general version of the \QAP \ was considered by Lawler \cite{Lawler1963}, who introduced a four-index cost array\footnote{We denoted the $4$-dimensional array $\mathsf K$ by sans-serif font in order to distinguish it from $2$-dimensional matrices.} $\mathsf K = (k_{irjs})$ instead of the three matrices $\bm F$, $\bm D$ and $\bm C$. The relationship between the three matrices and $\mathsf K$ is the following:
\begin{align}
\label{eq:defK}
k_{irjs}&=f_{ij}d_{rs} \quad &\text{for $i \neq j$ or $r \neq s$},\\
k_{irir}&=f_{ii}d_{rr} +c_{ir} \quad &\text{for $i,r\in \{1,\dots,n\}$}.
\end{align}


\noindent Moreover, note that the objective function can be re-written as follows:
\small
\[
\begin{split}
\sum_{i=1}^n
\left(
 f_{ii}d_{\pi(i)\pi(i)}
 + c_{i\pi(i)} 
 + \sum_{\substack{j=1\\j\neq i}}^n f_{ij}d_{\pi(i)\pi(j)}
 \right) 
 &= \sum_{i=1}^n
 \left(k_{i \pi(i) i \pi(i)}
 + \sum_{\substack{j=1\\j\neq i}}^n k_{i \pi(i) j \pi(j)}
 \right) \\
 &= \sum_{i=1}^n
 \left(\,\sum_{j=1}^n k_{i \pi(i) j \pi(j)}
 \right)
 .
 \end{split}
\]
\normalsize
According to this notation, the general form of the \QAP \ is
\begin{tcolorbox}[title=Lawler's general formulation]
	\begin{equation}
	\label{eq:QAPgenerale}
	\min_{\pi \in \Sn} \left[ \sum_{i=1}^n\sum_{j=1}^n k_{i \pi(i) j \pi(j)}\right]
	\end{equation}
\end{tcolorbox}



The combinatorial formulation \eqref{eq:defQAP} is the most known, but there are others, totally equivalent, that we are going to describe.
\subsection{Algebraic formulation}
An other formulation is the algebraic one, which contains binary variables.

Let $x_{ir}$ such that
\[
x_{ir}=\begin{cases}
1 & \text{if the facility $i$ is assigned to location $r$}\\
0 & \text{otherwise.}
\end{cases}\]
Hence the \QAP \ can be formulated as
\begin{tcolorbox}[title=Algebraic formulation]
\begin{equation}
\label{eq:defQAP2}
\begin{split}
&\min{\left[\sum_{i=1}^n\sum_{j=1}^n\sum_{r=1}^n\sum_{s=1}^nf_{ij}d_{rs}x_{ir}x_{js}+\sum_{i=1}^n\sum_{r=1}^n c_{ir} x_{ir}\right]}\\
&\text{ s.t} \quad  \sum_{i=1}^n x_{ir} = 1 \qquad \forall r\in \{1,\dots,n\}\\
&\phantom{\text{ s.a}} \quad  \sum_{r=1}^n x_{ir} = 1 \qquad \forall i\in \{1,\dots,n\}\\
&\phantom{\text{ s.a}} \quad  x_{ir}\in \{0,1\} \qquad \forall i,j\in \{1,\dots,n\} \\
\end{split}
\end{equation}
\end{tcolorbox}

Using Lawler's general form \eqref{eq:QAPgenerale} this formulation can be written as
\begin{tcolorbox}[title=Algebraic formulation in Lawler's form]
\begin{equation}
\label{eq:defQAP2Generale}
\begin{split}
&\min{\left[\sum_{i=1}^n\sum_{j=1}^n\sum_{r=1}^n\sum_{s=1}^nk_{irjs}x_{ir}x_{js}\right]}\\
&\text{ s.t} \quad  \sum_{i=1}^n x_{ir} = 1 \qquad \forall r\in \{1,\dots,n\}\\
&\phantom{\text{ s.a}} \quad  \sum_{r=1}^n x_{ir} = 1 \qquad \forall i\in \{1,\dots,n\}\\
&\phantom{\text{ s.a}} \quad  x_{ir}\in \{0,1\} \qquad \forall i,j\in \{1,\dots,n\}\\
\end{split}
\end{equation}
\end{tcolorbox}


\subsection{Inner product formulation}
The combinatorial formulation \eqref{eq:defQAP} can be written in a more compact way using the inner product of permutation matrices.
\begin{defi}[Permutation Matrix]
	Let $n\in \N$ and $\bm P$ be an $n \times n$ binary matrix. $\bm P$ is called a \textit{permutation matrix} if 
	\small
	\[
	\sum_{i=1}^n p_{ij}=1 \text{ for every $j=1,\dots,n$}\quad\text{and} \quad \sum_{j=1}^n p_{ij}=1 \text{ for every $i=1,\dots,n$} \footnote{Hence the matrix $\bm X = (x_{ir})$ in equation \eqref{eq:defQAP2} is a permutation matrix.}.
	\]	
\end{defi}
\normalsize
\noindent Therefore, every permutation matrix is a matrix with one (and only one)~$1$ for every row and column. This suggest its name: for every permutation $\pi \in \Sn$ exists one and only one permutation matrix $\bm X_\pi= (x_{ir})$ such that:
\[ x_{ir}=\begin{cases}
1 & \text{if $\pi(i) = r$;} \\
0 & \text{otherwise.}
\end{cases}\]




\noindent For example, if $n=4$ the permutation $\pi = [4,2,3,1]$ is associated to the permutation matrix:
\begin{equation}
	\label{eq:matricePermu}
\bm X_\pi = \begin{pmatrix}
0 & 0 & 0 & 1 \\
0 & 1 & 0 & 0 \\
0 & 0 & 1 & 0 \\
1 & 0 & 0 & 0 \\
\end{pmatrix}.
\end{equation}

\noindent Hence, given a vector $\bm v\in \R^n$, the vector $\bm w :=\bm X_\pi \cdot \bm v$ is  obtained permuting the entries of $\bm v$ according to the permutation $\pi$, i.e., \[w_i ~=~v_{\pi(i)} \quad \forall i\in \{1,\dots,n\}.\]

\noindent In fact, if we consider the permutation matrix described in \eqref{eq:matricePermu}, we obtain

\[
\bm X_\pi \cdot \bm v =
 \begin{pmatrix}
	0 & 0 & 0 & 1 \\
	0 & 1 & 0 & 0 \\
	0 & 0 & 1 & 0 \\
	1 & 0 & 0 & 0 \\
\end{pmatrix} 
\begin{pmatrix}
v_1\\
v_2\\
v_3\\
v_4\\	
\end{pmatrix}
= \begin{pmatrix}
	v_4\\
	v_2\\
	v_3\\
	v_1\\	
\end{pmatrix}.
\]


\noindent If $\bm A$ and $\bm B$ are two matrices $n \times n$, we can define their inner product $ \langle \bm A, \bm B \rangle $.


\begin{defi}[Inner product of two matrices]
	Let $\bm A= (a_{ij})$ and $\bm B=(b_{ij})$ be two real matrices $n \times n$. We define the \textit{inner product}\footnote{Sometimes in literature this operation is called \textit{Hadamard product}, from the French mathematician Jacques Hadamard.} of $\bm A$ and $\bm B$ as the real number defined by
\[
\langle \bm A,\bm B \rangle 
:= \sum_{r=1}^n \sum_{s=1}^n a_{rs}b_{rs}.
\]
\end{defi}

\noindent Letting $\bm X_\pi$ be a permutation matrix and $x_{ij}$ its entry $(i,j)$, we note that 
\begin{equation}
	\label{eq:MatPermQAP}
	\begin{split}
\Bigl(\bm X_\pi \,   \bm D \, \bm {X}_\pi \trasp \, \Bigr)_{ij} 
&=\sum_{r=1}^n x_{ir} \big( \bm D \, \bm {X}_\pi \trasp \big)_{rj}   \\
&=\sum_{r=1}^n x_{ir} \left(\sum_{s=1}^n d_{rs}x_{js}\right)   \\
&=\sum_{r=1}^n \sum_{s=1}^n x_{ir}  d_{rs}x_{js}.
\end{split}
\end{equation}
By \eqref{eq:defQAP2} and \eqref{eq:MatPermQAP} it follows that the \QAP \ can be rewritten as 

\begin{tcolorbox}[title=Inner product formulation]
\begin{equation}
\label{eq:defQAP3}
\begin{split}	
&\min{\left[\langle \bm F, \bm X \bm D \bm X\trasp \rangle +  \langle \bm C, \bm X \rangle \right]} \\
&\text{ s.t} \quad \bm X \text{ is a permutation matrix} \\
\end{split}
\end{equation}
\end{tcolorbox}


\subsection{Trace formulation}

\begin{defi}[Trace of a Matrix]
	Let $\bm A=(a_{ij})_{ij}$ be a matrix $n\times n$. We define \textit{trace} of $\bm A$ as the real number given by the sum of its diagonal elements:
	\[
	\tr{\bm A} = \sum_{i=1}^n a_{ii}
	\]
\end{defi}

\begin{prop}
Let  $\bm A$, $\bm B$ be two  $n \times n$ matrices, then we get some simple properties of the trace:
\begin{enumerate}
	\item $\tr{(\bm A + \bm B)}= \tr \bm A + \tr \bm B$; \label{propr:traccia1}
	\item $\tr{ \bm A\trasp} = \tr \bm A$;\label{propr:traccia2}
	\item $\tr{(\bm A \bm B)} = \tr{(\bm A \trasp \bm B \trasp )}$;\label{propr:traccia3}
	\item $\langle \bm A,\bm B \rangle = \tr{(\bm A\trasp \bm B)}$.\label{propr:traccia4}
\end{enumerate}
\end{prop}
We can rewrite  \eqref{eq:defQAP3} using the trace operator, in fact
\[
\begin{split}
\langle \bm F, \bm X \bm D \bm X\trasp \rangle +  \langle \bm C, \bm X \rangle &\overset{\ref{propr:traccia4}}{=}  \tr{ \left(\bm F \trasp  \bm X \bm B \bm X\trasp\right)} + \tr{\left(\bm C\trasp \bm  X\right)} \\
 &\overset{\ref{propr:traccia3}}{=} \tr{ \left(\bm F   \bm X\bm D\trasp \bm X\trasp\right)} +\tr{\left(\bm C \bm  X\trasp \right)}\\
 &\overset{\ref{propr:traccia1}}{=} \tr{\left(\bm F   \bm X \bm D \trasp \bm X\trasp + \bm C \bm  X \trasp \right)} \\
 &= \tr{\Bigl((\bm F  \bm X \bm D \trasp  + \bm C)\bm X \trasp \Bigr)}
 \end{split}
\]
Therefore the equation \eqref{eq:defQAP3} can be rewritten as
\begin{tcolorbox}[title=Trace formulation]
	\begin{equation}
	\label{eq:QAP_traccia}
	\begin{split}	
	&\min{\left[ \tr{\Bigl((\bm F \trasp  \bm X \bm D + \bm C)\bm X \trasp \Bigr)} \right]} \\
	&\text{ s.t.} \quad \bm X \text{ is a permutation matrix.} \\
	\end{split}
	\end{equation}
\end{tcolorbox}

The trace formulation of the QAP was used by Finke, Burkard e Rendl \cite{Finke1987} to introduce eigenvalue bounds for QAP.


\subsection{Kronecker product formulation}
A further reformulation can be observed exploiting the Kronecker product of two matrices.
\begin{defi}[Kronecker product]
	Let $\bm A \in \R^{m \times n}$ and $\bm B \in \R^{r \times s}$ two matrices. We define the Kronecker product $\bm A \otimes \bm B \in \R^{mr \times ns}$ as the matrix formed by all possible products $a_{ij}b_{hk}$:
\[
\bm A \otimes \bm B = \begin{pmatrix}
a_{11}B & a_{12} B & \dots & a_{1n}B \\
\vdots & \vdots & \ddots & \vdots \\
a_{m1}B& a_{m2}B & \dots & a_{mn}B \\
\end{pmatrix}.
\]	
Now, let $\bm X$ be a permutation matrix. We can consider the four-index cost array $\mathsf K$ introduced in equation \eqref{eq:defK}, as \virgolette{matrix of matrices}, so every $n \times n$ matrix $\bm K^{ir}$ is formed by the elements $k_{irjs}$	with fixed indices $i$ and $r$ and variable indices $j,s = 1,2,\dots,n$. Using this notation we get

\[
\begin{split}
\langle \mathsf K, \bm X \otimes \bm X \rangle &=\left\langle \begin{bmatrix}
\bm K^{11} & \dots &\bm  K^{1n} \\
\vdots & \ddots & \vdots \\
\bm K^{n1} & \dots & \bm K^{nn} \\
\end{bmatrix}
\begin{bmatrix}
x_{11}\bm X &\dots &x_{1n}\bm X \\
\vdots & \ddots & \vdots \\
x_{n1}\bm X & \dots & x_{nn}\bm X \\
\end{bmatrix}\right\rangle \\
&=\sum_{i=1}^n\sum_{r=1}^n x_{ir}\langle \bm K^{ir},\bm X  \rangle \\
&=\sum_{i=1}^n\sum_{r=1}^n \sum_{j=1}^n\sum_{s=1}^n k_{irjs}x_{ir}x_{js},
\end{split}
\]
which is the objective function from \eqref{eq:defQAP2Generale}.
	This lead us to the Kronecker product formulation of the QAP.
	
	\begin{tcolorbox}[title = Kronecker product formulation]
		\begin{equation}
		\begin{split}	
		&\min{\langle \mathsf K, \mathsf Y \rangle} \\
		&\text{ s.t.} \quad   \mathsf Y = \bm X \otimes \bm X \\
		&\phantom{\text{ s.t.}} \quad \bm X \text{ is a permutation matrix} \\
\end{split}
		\end{equation}
	\end{tcolorbox}
	
\end{defi}

\begin{comment}
\subsection{Convex and concave integer formulation}
For any $\bm A \in \R^{n \times n}$ let $\vettore(\bm A) \in \R^{n^2}$ be the vector obtained by stacking the columns of $\bm A$ on top of one other.

Using  $\tr(\bm A \bm B) = \tr{(\bm B\trasp \bm A \trasp )} $  and $\tr\left(\bm A \bm B\trasp \right) = \vettore(\bm A\trasp) \vettore(\bm B) $, the trace formulation \eqref{eq:QAP_traccia} can be rewritten as
\[
\begin{split}
\tr{\Bigl((\bm F  \bm X \bm D  \trasp+ \bm C)\bm X \trasp \Bigr)} 
&=\tr{\Bigl(\bm F   \bm X \bm D\trasp \bm X \trasp \Bigr)} + \tr{\left(\bm C \bm X \trasp \right)} \\
&=\tr{\Bigl(\bm F \trasp  \bm X \bm D \bm X \trasp \Bigr)} + \tr{\left(\bm X \bm C \trasp \right)} \\
&=\vettore(\bm X) \trasp\vettore{\left(\bm F   \bm X \bm D\trasp \right)} +\vettore(\bm C) \trasp \vettore(\bm X)  \\
&=\vettore(\bm X) \trasp \left(\bm D  \otimes \bm F\right)\vettore(\bm X) +\vettore(\bm C) \trasp \vettore(\bm X)
\end{split}
\]


Were we used in the final step
\[
\vettore{(\bm F \bm X \bm V)} = \left(\bm V\trasp \otimes \bm A\right) \vettore(\bm X)
\]

(See theorem 2, p. 35 in \textit{Matrix Differential Calculus with Applications in Statistics and Econometrics} by J. R. Magnus and H. Neudecker \cite{Magnus1999}).


The element $f_{ik}d_{jl}$ lies in the $((j-1)n+i)$th row and $((l-1)n+k)$th column of matrix $\bm D \otimes \bm F$. Therefore, we can arrange the $n^4$ cost coefficients $d_{ijkl}$ in a new way so that the element $d_{ijkl}$ lies in row $(j-1)n+i$ and column $(l-1)n+k$ of an $n^2\times n^2$ matrix $\bm M$.
The general QAP can be written as

\begin{tcolorbox}[title=Convex formulation]
	\begin{equation}
\begin{split}	
&\min{\left[\bm x\trasp \bm M \bm x\right]} \\
&\text{ s.t.} \quad   \bm X \text{ is a permutation matrix}\\
&\phantom{\text{ s.t.}} \quad \bm x = \vettore(\bm X)
\end{split}
	\end{equation}
\end{tcolorbox}

We can assume that $\bm M$ is symmetric and positive definite/negative, so we can write a \QAP \ as a quadratic convex program or quadratic concave program.


\end{comment}

\section{Variants}
\label{sec:Variants}
There are some variants in literature. We are going to show some of the principal ones. Many others can be found in \cite[Ch. 9]{Burkard2012}.
\subsection{QBAP}
The most known variants of the \QAP \ is the \textit{Quadratic Bottleneck Assignment Problem} (QBAP), that is obtained by replacing the sums in the objective function of a QAP with the maximum operator.

The QBAP can be formulated as

\begin{equation}
\min_{\pi \in S_n} {\left[\max{\left(\max_{1\le i,j \le n} {f_{ij}d_{\pi{(i)}\pi{(j)}}}, \max_{1\le i\le n}{c_{i\pi(i)}}\right)}\right]}. 
\end{equation}
Basically all QAP applications give rise to a QBAP model as well, because it often makes sense to minimize the largest cost instead of the overall cost incurred by some decisions.

\subsection{Quadratic semi-assignment problem}
The \textit{quadratic semi-assignment problem} (semi-\QAP) has the same objective function as the \QAP \ but allows the solution not to be permutations: they map the set of integers $ N=\{1,2,\dots,n\}$ to the set $ M=\{1,2,\dots,m\}$ with $n>m$ (so there are more facilities than locations and there is no limit to the number of facilities we can assign to the same location)

We can write the semi-\QAP \ in the algebraic form as
	\begin{equation}
	\label{eq:Semi-qap}
	\begin{split}
	&\min{\left[\sum_{i=1}^n\sum_{j=1}^n\sum_{r=1}^m\sum_{s=1}^mf_{ij}d_{rs}x_{ir}x_{js}+\sum_{i=1}^n\sum_{r=1}^m c_{ir} x_{ir}\right]}\\
	&\text{ s.t} \quad  \sum_{r=1}^m x_{ir} = 1 \quad i=1,\dots,n \\
	&\phantom{\text{ s.t}} \quad  x_{ir}\in \{0,1\} \quad i = 1,2,\dots,n \quad j=1,2,\dots,m\\
	\end{split}
	\end{equation}

\noindent The semi-\QAP \ is $\mathbf{NP}$-hard, as an instance of QAP$(\bm F, \bm D, \bm C)$ can always be transformed into an equivalent instance of semi-QAP by adding slack variables.



From now on, the matrix $\bm C$ will be the null matrix, hence, no linear term will appear in the objective function.