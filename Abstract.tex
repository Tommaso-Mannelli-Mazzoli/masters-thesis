% !TeX spellcheck = it_IT
% !TEX encoding = UTF-8 Unicode
\documentclass[a4paper,openright, oneside]{scrartcl}
%\documentclass[a4paper]{amsart}
\usepackage[utf8]{inputenc}
\usepackage{type1ec}
\usepackage[T1]{fontenc}
\usepackage{pdfx}
%\usepackage{lmodern}
\usepackage[italian]{babel}			   
			   
%\usepackage{classicthesis}			   
\usepackage[
%	pdfpagemode={UseOutlines},%
%	bookmarksopen,%%
%	pdfstartview={FitH},%
%	colorlinks
]{hyperref}
\date{}
\title{Abstract}
\begin{document}
\maketitle
\begin{description}
	\item[Candidato] Tommaso Mannelli Mazzoli.
	\item[Titolo della tesi] Il Problema di Assegnazione Quadratica: approcci metaeuristici.
	\item[Relatore] prof.ssa Stefania Bellavia\footnote{\url{stefania.bellavia@unifi.it}, Università degli Studi di Firenze, Firenze.}.
	\item[Correlatore] prof. Angel Felipe Ortega\footnote{\url{felipe@mat.ucm.es}, Universidad Complutense de Madrid, Madrid.}.
\end{description}


Nella tesi si studia il Problema di Assegnazione Quadratica (QAP). Il QAP è un problema di ottimizzazione combinatoria, che è stato dimostrato appartenere alla classe di problemi $\mathbf{NP}$-hard.

Lo scopo di questa tesi è quello di descrivere il problema, alcune sue formulazioni e applicazioni, studiare vari algoritmi euristici e metaeuristici, implementarli e compararli.

Per quanto riguarda i metodi euristici abbiamo implementato algoritmi greedy e di ricerca locale.

Questi metodi sono stati usati per costruire algoritmi metaeuristici più avanzati, in grado di fornire soluzioni migliori. Abbiamo studiato e implementato l'Algoritmo di Colonia di Formiche (Ant Colony Optimization), la Ricerca Tabu (Tabu Search) e la Ricerca in Intorno Variabile (Variable Neighborhood Search).

I metodi sono stato implementati in linguaggio Fortran e i codici sono stati resi disponibili nella repository di GitHub.

Alla fine del documento vengono mostrati risultati numerici ottenuti comparando gli algoritmi su varie istanze presenti sulla libreria QAPLIB.
\end{document}